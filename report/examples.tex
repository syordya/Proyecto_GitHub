\documentclass[10pt,a4paper]{amsart}
% \documentclass[10pt,a4paper]{article}
\usepackage[utf8x]{inputenc}
\usepackage{ucs}
\usepackage[brazil]{babel}
\usepackage[T1]{fontenc}
\usepackage{amsmath}
\usepackage{amsfonts}
\usepackage{amssymb}
\usepackage{indentfirst}
%to bind figures in respective sections
\usepackage[section]{placeins}
%to number figures considering the section
\usepackage{chngcntr}
\counterwithin{figure}{section}

%for pseudocode
% \usepackage[]{algorithm2e}
\usepackage{algorithm}
\usepackage[noend]{algpseudocode}
\makeatletter
\def\BState{\State\hskip-\ALG@thistlm}
\makeatother
\floatname{algorithm}{Algoritmo}
\renewcommand{\algorithmicrequire}{\textbf{Entrada:}}
\renewcommand{\algorithmicensure}{\textbf{Saída:}}
\renewcommand{\algorithmicwhile}{\textbf{Enquanto}}
\renewcommand{\algorithmicdo}{\textbf{faça}}
\renewcommand{\algorithmicfor}{\textbf{Para}}
\renewcommand{\algorithmicif}{\textbf{Se}}
\renewcommand{\algorithmicthen}{\textbf{então}}

% for pretty-printed source code
\usepackage{listings}
% environment for C code
\lstnewenvironment{code}
 {\lstset{ %
     extendedchars=true,
     stringstyle=\ttfamily \scriptsize, %
     showstringspaces=false, aboveskip={1.\baselineskip}, %
     identifierstyle=\ttfamily \scriptsize \bf, %
     language=C,           %
     basicstyle=\ttfamily \small,  %\footnotesize
     numberstyle=\footnotesize, %
     % keywordstyle=\bf,       % keyword style
     tabsize=1,                 % sets default tabsize to 2 spaces
     captionpos=t,              % sets the caption-position to bottom
     breaklines=true,           % sets automatic line breaking
     breakatwhitespace=false,   % sets if automatic breaks should only happen at whitespace
     %backgroundcolor=\color{black!10},
   }
} {}

\author{Jo\~ao S. Brito Jr.\\NUSP: 5889672}
\title{Laboratório de Inteligência Artificial: \\Exemplos para testes}
\date{\today}
\usepackage{graphicx}
\graphicspath{ {images/} }

\begin{document}

% \begin{abstract}
%   % Resumo
%   Este documento descreve...
% \end{abstract}

\maketitle

\section{Exemplo 1}
\subsection{Base}
\begin{itemize}
 \item Todos os pássaros botam ovos
 \item Todo animal com asas pode voar
 \item Todo animal com asas é um pássaro
 \item Eu vi um animal com asas
 \item O animal que vi era um morcego
\end{itemize}
\subsection{Consequências}
\begin{itemize}
 \item Todos os pássaros voam
 \item Morcegos são pássaros
 \item Morcegos voam
 \item Morcegos põem ovos
\end{itemize}
\subsection{Nova Informação}
\begin{itemize}
 \item Morcegos não são pássaros
 \item O animal que vi era um pinguim
\end{itemize}

\section{Exemplo 2}
\subsection{Base}
\begin{itemize}
 \item Um problema na internet pode ser um problema no roteador, ou um problema na linha telefônica, ou um problema interno da operadora
 \item A internet está com problema
 \item O telefone não está com problema
\end{itemize}
\subsection{Consequências}
\begin{itemize}
 \item Há um problema no roteador ou na operadora
\end{itemize}
\subsection{Nova Informação}
\begin{itemize}
 \item O roteador não está com problema
\end{itemize}

\end{document}
